\section{Introducción}

En el presente trabajo se aborda el diseño de una línea de transmisión bifilar, un tipo de conductor eléctrico compuesto por dos hilos paralelos separados por un dieléctrico. Este tipo de línea es ampliamente utilizado en diversas aplicaciones de telecomunicaciones debido a sus características particulares.

El objetivo principal de este estudio es desarrollar un diseño que cumpla con un conjunto específico de requisitos, incluyendo una impedancia característica de $300 \Omega$, una capacitancia por unidad de longitud de $10 pF/m$, una atenuación nominal de $1.9 dB/m$ a $100 MHz$ y una velocidad de propagación del $98 \%$ de la velocidad de la luz en el vacío.

La importancia de este diseño radica en la necesidad de transmitir señales electromagnéticas de manera eficiente y con mínimas pérdidas en aplicaciones donde estas especificaciones son fundamentales. A través de este trabajo, se busca analizar los factores que influyen en el rendimiento de una línea de transmisión bifilar y proponer una solución óptima que satisfaga los requerimientos establecidos.

En las siguientes secciones se presentarán los fundamentos teóricos de las líneas de transmisión, el procedimiento de diseño seguido y los resultados obtenidos. Finalmente, se discutirán las posibles aplicaciones de este diseño y se propondrán líneas futuras de investigación.


\pagebreak